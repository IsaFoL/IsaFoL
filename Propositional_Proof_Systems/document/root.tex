\documentclass[11pt,a4paper]{article}
\usepackage{isabelle,isabellesym}

\usepackage[x11names, rgb]{xcolor}
\usepackage[utf8]{inputenc}
\usepackage{tikz}
\usetikzlibrary{snakes,arrows,shapes}
\usepackage{amsmath}
\DeclareUnicodeCharacter{2192}{$\rightarrow$}
\DeclareUnicodeCharacter{2194}{$\leftrightarrow$}

% further packages required for unusual symbols (see also
% isabellesym.sty), use only when needed

\usepackage{amssymb}
  %for \<leadsto>, \<box>, \<diamond>, \<sqsupset>, \<mho>, \<Join>,
  %\<lhd>, \<lesssim>, \<greatersim>, \<lessapprox>, \<greaterapprox>,
  %\<triangleq>, \<yen>, \<lozenge>

%\usepackage{eurosym}
  %for \<euro>

%\usepackage[only,bigsqcap]{stmaryrd}
  %for \<Sqinter>

%\usepackage{eufrak}
  %for \<AA> ... \<ZZ>, \<aa> ... \<zz> (also included in amssymb)

%\usepackage{textcomp}
  %for \<onequarter>, \<onehalf>, \<threequarters>, \<degree>, \<cent>,
  %\<currency>

%\renewcommand{\isasymdots}{\dots}

% this should be the last package used
\usepackage{pdfsetup}

% urls in roman style, theory text in math-similar italics
\urlstyle{rm}
\isabellestyle{it}

% for uniform font size
%\renewcommand{\isastyle}{\isastyleminor}

\begin{document}

\title{Propositional Proof Systems}
\author{Julius Michaelis and Tobias Nipkow}
\maketitle
\begin{abstract}
We present a formalization of Sequent Calculus, Natural Deduction, Hilbert Calculus, and Resolution using a deep embedding of propositional formulas. 
We provide proofs of many of the classical results, including Cut Elimination, Craig's Interpolation, proof transformation between all calculi, and soundness and completeness. 
\end{abstract}

\tableofcontents
\vspace{1em}

The files of this entry are organized in a way that should allow loading only that part of the formalization that the user is interested in.
Special care was taken not to mix proofs that require semanics and proofs that talk about transformation between proof systems.
An overview of the different theory files and their dependencies can be found in figures \ref{fig:prooftran} and \ref{fig:sema}.

% ./overview.sh tex sema >document/fig_sema.tex && ./overview.sh tex prooftran >document/fig_tran.tex
\begin{figure}
	\centering
	
\begin{tikzpicture}[>=latex,line join=bevel,scale=0.16, every node/.style={scale=0.3}]
  \pgfsetlinewidth{1bp}
%%
\pgfsetcolor{black}
  % Edge: ND -> Formulas
  \draw [->,-stealth,very thin] (309.28bp,269.98bp) .. controls (361.17bp,291.77bp) and (458.45bp,331.06bp)  .. (543.99bp,356.9bp) .. controls (597.66bp,373.12bp) and (659.57bp,387.08bp)  .. (716.06bp,398.7bp);
  % Edge: MiniSC_HC -> MiniSC
  \draw [->,-stealth,very thin] (875.99bp,173.44bp) .. controls (875.99bp,184.06bp) and (875.99bp,198.04bp)  .. (875.99bp,221.62bp);
  % Edge: SCND -> ND
  \draw [->,-stealth,very thin] (268.73bp,173.44bp) .. controls (269.32bp,187.69bp) and (270.16bp,207.98bp)  .. (271.27bp,234.48bp);
  % Edge: MiniSC_HC -> HC
  \draw [->,-stealth,very thin] (829.66bp,170.6bp) .. controls (812.85bp,175.75bp) and (793.64bp,181.37bp)  .. (775.99bp,185.91bp) .. controls (702.04bp,204.93bp) and (681.75bp,202.12bp)  .. (607.99bp,221.91bp) .. controls (594.8bp,225.45bp) and (580.71bp,229.74bp)  .. (557.78bp,237.15bp);
  % Edge: SC_Gentzen -> SC_Cut
  \draw [->,-stealth,very thin] (676.55bp,173.44bp) .. controls (676.88bp,184.06bp) and (677.31bp,198.04bp)  .. (678.05bp,221.62bp);
  % Edge: MiniSC -> MiniFormulas
  \draw [->,-stealth,very thin] (875.63bp,284.16bp) .. controls (875.53bp,292.48bp) and (875.43bp,301.46bp)  .. (875.21bp,319.81bp);
  % Edge: LSC_Resolution -> Resolution
  \draw [->,-stealth,very thin] (1620.5bp,62.346bp) .. controls (1637.8bp,80.588bp) and (1664.9bp,109.3bp)  .. (1691.2bp,137.05bp);
  % Edge: NDHC -> HC
  \draw [->,-stealth,very thin] (411.75bp,171.9bp) .. controls (431.89bp,187.53bp) and (462.66bp,211.41bp)  .. (493.34bp,235.21bp);
  % Edge: SCND -> SC
  \draw [->,-stealth,very thin] (296.77bp,170.28bp) .. controls (319.16bp,182.34bp) and (350.31bp,200.89bp)  .. (373.99bp,221.91bp) .. controls (400.53bp,245.45bp) and (393.74bp,265.6bp)  .. (423.99bp,284.13bp) .. controls (471.54bp,313.26bp) and (532.38bp,326.73bp)  .. (592.77bp,333.99bp);
  % Edge: LSC -> SC_Cut
  \draw [->,-stealth,very thin] (1412.9bp,177.36bp) .. controls (1396.7bp,180.68bp) and (1379.9bp,183.72bp)  .. (1364.0bp,185.91bp) .. controls (1094.1bp,223.06bp) and (1017.6bp,161.12bp)  .. (751.99bp,221.91bp) .. controls (745.77bp,223.33bp) and (739.39bp,225.24bp)  .. (723.4bp,230.97bp);
  % Edge: MiniSC_Craig -> MiniSC
  \draw [->,-stealth,very thin] (1187.3bp,170.62bp) .. controls (1128.6bp,185.97bp) and (1036.0bp,210.19bp)  .. (957.01bp,230.84bp);
  % Edge: MiniFormulas -> Formulas
  \draw [->,-stealth,very thin] (851.77bp,356.29bp) .. controls (838.93bp,365.56bp) and (822.78bp,377.22bp)  .. (800.39bp,393.39bp);
  % Edge: LSC -> CNF
  \draw [->,-stealth,very thin] (1502.0bp,185.98bp) .. controls (1502.0bp,198.21bp) and (1502.0bp,212.26bp)  .. (1502.0bp,234.55bp);
  % Edge: SC_Cut -> SC
  \draw [->,-stealth,very thin] (679.36bp,284.16bp) .. controls (679.46bp,292.48bp) and (679.56bp,301.46bp)  .. (679.78bp,319.81bp);
  % Edge: SC -> Formulas
  \draw [->,-stealth,very thin] (703.48bp,356.65bp) .. controls (716.2bp,365.93bp) and (732.08bp,377.52bp)  .. (754.06bp,393.56bp);
  % Edge: HCSC -> SC_Cut
  \draw [->,-stealth,very thin] (539.78bp,170.83bp) .. controls (563.93bp,184.91bp) and (600.64bp,206.32bp)  .. (639.26bp,228.85bp);
  % Edge: HCSCND -> SCND
  \draw [->,-stealth,very thin] (346.24bp,84.483bp) .. controls (328.69bp,100.03bp) and (309.18bp,117.32bp)  .. (286.53bp,137.37bp);
  % Edge: HCSCND -> HCSC
  \draw [->,-stealth,very thin] (435.75bp,84.483bp) .. controls (453.3bp,100.03bp) and (472.81bp,117.32bp)  .. (495.46bp,137.37bp);
  % Edge: MiniSC_Craig -> Formulas
  \draw [->,-stealth,very thin] (1225.6bp,173.06bp) .. controls (1183.7bp,211.63bp) and (1076.2bp,305.41bp)  .. (969.99bp,356.9bp) .. controls (931.95bp,375.35bp) and (886.49bp,388.32bp)  .. (839.47bp,399.03bp);
  % Edge: NDHC -> ND
  \draw [->,-stealth,very thin] (371.08bp,171.9bp) .. controls (351.95bp,187.37bp) and (322.84bp,210.91bp)  .. (293.18bp,234.89bp);
  % Edge: Resolution -> CNF
  \draw [->,-stealth,very thin] (1675.9bp,170.41bp) .. controls (1638.2bp,188.1bp) and (1574.9bp,217.79bp)  .. (1526.7bp,240.45bp);
  % Edge: MiniSC_Cut -> MiniSC
  \draw [->,-stealth,very thin] (1008.3bp,178.84bp) .. controls (986.11bp,191.12bp) and (958.44bp,206.43bp)  .. (925.37bp,224.71bp);
  % Edge: HCSC -> HC
  \draw [->,-stealth,very thin] (514.18bp,173.44bp) .. controls (514.33bp,187.69bp) and (514.54bp,207.98bp)  .. (514.81bp,234.48bp);
  % Edge: CNF -> Formulas
  \draw [->,-stealth,very thin] (1474.2bp,263.48bp) .. controls (1414.6bp,283.42bp) and (1270.2bp,330.1bp)  .. (1146.0bp,356.9bp) .. controls (1049.2bp,377.8bp) and (936.21bp,392.79bp)  .. (850.69bp,402.65bp);
  % Edge: ND_FiniteAssms -> ND
  \draw [->,-stealth,very thin] (146.73bp,182.34bp) .. controls (173.98bp,197.5bp) and (207.72bp,216.27bp)  .. (242.34bp,235.52bp);
  % Edge: LSC_Resolution -> LSC
  \draw [->,-stealth,very thin] (1587.6bp,62.346bp) .. controls (1574.0bp,76.873bp) and (1554.2bp,98.041bp)  .. (1529.9bp,124.0bp);
  % Edge: HCSCND -> NDHC
  \draw [->,-stealth,very thin] (390.99bp,87.902bp) .. controls (390.99bp,100.7bp) and (390.99bp,114.39bp)  .. (390.99bp,136.28bp);
  % Edge: MiniSC -> SC
  \draw [->,-stealth,very thin] (815.86bp,279.64bp) .. controls (787.19bp,291.85bp) and (753.42bp,306.24bp)  .. (717.23bp,321.66bp);
  % Edge: HC -> Formulas
  \draw [->,-stealth,very thin] (521.48bp,271.41bp) .. controls (530.98bp,294.46bp) and (550.89bp,334.8bp)  .. (580.99bp,356.9bp) .. controls (613.22bp,380.55bp) and (655.33bp,393.73bp)  .. (702.52bp,402.98bp);
  % Node: SCND
\begin{scope}
  \definecolor{strokecol}{rgb}{0.0,0.0,0.0};
  \pgfsetstrokecolor{strokecol}
  \draw (267.99bp,154.79bp) node {SC→ND};
\end{scope}
  % Node: Formulas
\begin{scope}
  \definecolor{strokecol}{rgb}{0.0,0.0,0.0};
  \pgfsetstrokecolor{strokecol}
  \draw (776.99bp,411.29bp) node {Formula Syntax};
\end{scope}
  % Node: HCSC
\begin{scope}
  \definecolor{strokecol}{rgb}{0.0,0.0,0.0};
  \pgfsetstrokecolor{strokecol}
  \draw (513.99bp,154.79bp) node {HC→SC};
\end{scope}
  % Node: ND_FiniteAssms
\begin{scope}
  \definecolor{strokecol}{rgb}{0.0,0.0,0.0};
  \pgfsetstrokecolor{strokecol}
  \draw (98.99bp,159.99bp) node {ND and a dead end};
  \draw (98.99bp,141.99bp) node {to compactness};
\end{scope}
  % Node: CNF
\begin{scope}
  \definecolor{strokecol}{rgb}{0.0,0.0,0.0};
  \pgfsetstrokecolor{strokecol}
  \draw (1502.0bp,253.02bp) node {CNF};
\end{scope}
  % Node: LSC
\begin{scope}
  \definecolor{strokecol}{rgb}{0.0,0.0,0.0};
  \pgfsetstrokecolor{strokecol}
  \draw (1501.99bp,159.99bp) node {Transforming SC: to CNF,};
  \draw (1501.99bp,141.99bp) node {to left handed SC (LSC)};
\end{scope}
  % Node: NDHC
\begin{scope}
  \definecolor{strokecol}{rgb}{0.0,0.0,0.0};
  \pgfsetstrokecolor{strokecol}
  \draw (390.99bp,154.79bp) node {ND→HC};
\end{scope}
  % Node: SC_Gentzen
\begin{scope}
  \definecolor{strokecol}{rgb}{0.0,0.0,0.0};
  \pgfsetstrokecolor{strokecol}
  \draw (675.99bp,154.79bp) node {Gentzen Style SC};
\end{scope}
  % Node: ND
\begin{scope}
  \definecolor{strokecol}{rgb}{0.0,0.0,0.0};
  \pgfsetstrokecolor{strokecol}
  \draw (271.99bp,253.02bp) node {Natural Deduction};
\end{scope}
  % Node: LSC_Resolution
\begin{scope}
  \definecolor{strokecol}{rgb}{0.0,0.0,0.0};
  \pgfsetstrokecolor{strokecol}
  \draw (1604.0bp,43.841bp) node {LSC ↔ Resolution};
\end{scope}
  % Node: MiniSC_Cut
\begin{scope}
  \definecolor{strokecol}{rgb}{0.0,0.0,0.0};
  \pgfsetstrokecolor{strokecol}
  \draw (1049.99bp,159.99bp) node {SC: cut};
  \draw (1049.99bp,141.99bp) node { contraction};
\end{scope}
  % Node: MiniFormulas
\begin{scope}
  \definecolor{strokecol}{rgb}{0.0,0.0,0.0};
  \pgfsetstrokecolor{strokecol}
  \draw (874.99bp,338.52bp) node {→ only formulas};
\end{scope}
  % Node: MiniSC
\begin{scope}
  \definecolor{strokecol}{rgb}{0.0,0.0,0.0};
  \pgfsetstrokecolor{strokecol}
  \draw (875.99bp,258.22bp) node {→ only transformation:};
  \draw (875.99bp,240.22bp) node {SC invariant};
\end{scope}
  % Node: MiniSC_Craig
\begin{scope}
  \definecolor{strokecol}{rgb}{0.0,0.0,0.0};
  \pgfsetstrokecolor{strokecol}
  \draw (1244.0bp,154.79bp) node {Interpolation using SC};
\end{scope}
  % Node: HCSCND
\begin{scope}
  \definecolor{strokecol}{rgb}{0.0,0.0,0.0};
  \pgfsetstrokecolor{strokecol}
  \draw (390.99bp,58.04bp) node {Single Formula};
  \draw (390.99bp,40.04bp) node {provability equivalence};
  \draw (390.99bp,22.04bp) node {HC↔ND↔SC};
\end{scope}
  % Node: MiniSC_HC
\begin{scope}
  \definecolor{strokecol}{rgb}{0.0,0.0,0.0};
  \pgfsetstrokecolor{strokecol}
  \draw (875.99bp,154.79bp) node {→ only: SC→HC};
\end{scope}
  % Node: SC
\begin{scope}
  \definecolor{strokecol}{rgb}{0.0,0.0,0.0};
  \pgfsetstrokecolor{strokecol}
  \draw (679.99bp,338.52bp) node {Sequent Calculus};
\end{scope}
  % Node: HC
\begin{scope}
  \definecolor{strokecol}{rgb}{0.0,0.0,0.0};
  \pgfsetstrokecolor{strokecol}
  \draw (514.99bp,253.02bp) node {Hilbert Calculus};
\end{scope}
  % Node: Resolution
\begin{scope}
  \definecolor{strokecol}{rgb}{0.0,0.0,0.0};
  \pgfsetstrokecolor{strokecol}
  \draw (1707.0bp,154.79bp) node {Resolution};
\end{scope}
  % Node: SC_Cut
\begin{scope}
  \definecolor{strokecol}{rgb}{0.0,0.0,0.0};
  \pgfsetstrokecolor{strokecol}
  \draw (678.99bp,258.22bp) node {SC: cut};
  \draw (678.99bp,240.22bp) node {contraction};
\end{scope}
%
\end{tikzpicture}


	\caption{Overview of results considering Proof Transformation}
	\label{fig:prooftran}
\end{figure}
\begin{figure}
	\centering
	
\begin{tikzpicture}[>=latex,line join=bevel,scale=0.16, every node/.style={scale=0.3}]
  \pgfsetlinewidth{1bp}
%%
\pgfsetcolor{black}
  % Edge: Compactness -> Sema
  \draw [->,-stealth,very thin] (510.76bp,355.81bp) .. controls (526.47bp,365.17bp) and (547.47bp,376.24bp)  .. (567.6bp,382.36bp) .. controls (607.35bp,394.43bp) and (868.31bp,418.82bp)  .. (1002.9bp,430.87bp);
  % Edge: ND -> Formulas
  \draw [->,-stealth,very thin] (399.02bp,445.52bp) .. controls (540.37bp,459.02bp) and (829.8bp,486.66bp)  .. (983.7bp,501.36bp);
  % Edge: Resolution_Compl_SC_Small -> CNF_Sema
  \draw [->,-stealth,very thin] (1163.9bp,45.104bp) .. controls (1303.8bp,55.862bp) and (1464.7bp,73.422bp)  .. (1486.6bp,98.225bp) .. controls (1504.9bp,118.94bp) and (1501.5bp,137.15bp)  .. (1486.6bp,160.45bp) .. controls (1476.3bp,176.64bp) and (1460.8bp,188.92bp)  .. (1434.9bp,202.82bp);
  % Edge: Resolution_Compl_SC_Small -> LSC_Resolution
  \draw [->,-stealth,very thin] (905.91bp,61.769bp) .. controls (884.46bp,76.007bp) and (859.17bp,92.788bp)  .. (830.95bp,111.52bp);
  % Edge: Resolution_Sound -> Resolution
  \draw [->,-stealth,very thin] (1106.8bp,147.99bp) .. controls (1108.6bp,162.23bp) and (1111.1bp,182.52bp)  .. (1114.4bp,209.03bp);
  % Edge: SCND -> ND
  \draw [->,-stealth,very thin] (192.82bp,355.22bp) .. controls (204.57bp,363.43bp) and (219.27bp,373.57bp)  .. (232.6bp,382.36bp) .. controls (248.54bp,392.86bp) and (266.39bp,404.14bp)  .. (290.23bp,418.97bp);
  % Edge: CNF -> Formulas
  \draw [->,-stealth,very thin] (970.68bp,356.9bp) .. controls (970.21bp,380.73bp) and (972.28bp,424.14bp)  .. (990.6bp,455.13bp) .. controls (997.92bp,467.5bp) and (1009.3bp,478.08bp)  .. (1028.8bp,492.22bp);
  % Edge: ND_Compl_SC -> SC_Sema
  \draw [->,-stealth,very thin] (400.2bp,246.26bp) .. controls (448.36bp,258.54bp) and (512.74bp,276.0bp)  .. (568.6bp,294.68bp) .. controls (588.8bp,301.43bp) and (610.71bp,309.9bp)  .. (638.99bp,321.42bp);
  % Edge: SC_Compl_Consistent -> SC_Sema
  \draw [->,-stealth,very thin] (676.6bp,258.91bp) .. controls (676.6bp,274.87bp) and (676.6bp,294.3bp)  .. (676.6bp,319.9bp);
  % Edge: LSC_Resolution -> Resolution
  \draw [->,-stealth,very thin] (858.18bp,144.79bp) .. controls (876.36bp,149.77bp) and (896.87bp,155.37bp)  .. (915.6bp,160.45bp) .. controls (975.12bp,176.59bp) and (991.51bp,175.74bp)  .. (1049.6bp,196.45bp) .. controls (1058.6bp,199.66bp) and (1068.1bp,203.57bp)  .. (1086.3bp,211.72bp);
  % Edge: ND_Compl_Truthtable_Compact -> Compactness
  \draw [->,-stealth,very thin] (319.04bp,142.92bp) .. controls (356.04bp,152.28bp) and (398.05bp,168.48bp)  .. (427.6bp,196.45bp) .. controls (459.92bp,227.03bp) and (474.8bp,278.23bp)  .. (483.07bp,319.71bp);
  % Edge: Substitution_Sema -> Sema
  \draw [->,-stealth,very thin] (1557.2bp,355.92bp) .. controls (1536.6bp,365.07bp) and (1509.6bp,375.9bp)  .. (1484.6bp,382.36bp) .. controls (1467.3bp,386.82bp) and (1237.0bp,414.54bp)  .. (1112.4bp,429.37bp);
  % Edge: Resolution_Compl_SC_Small -> SC_Sema
  \draw [->,-stealth,very thin] (790.28bp,54.211bp) .. controls (708.24bp,75.25bp) and (617.77bp,116.31bp)  .. (579.6bp,196.45bp) .. controls (567.71bp,221.42bp) and (567.72bp,233.7bp)  .. (579.6bp,258.68bp) .. controls (591.19bp,283.02bp) and (614.38bp,302.31bp)  .. (643.9bp,321.09bp);
  % Edge: Resolution_Sound -> CNF_Sema
  \draw [->,-stealth,very thin] (1144.6bp,146.44bp) .. controls (1178.0bp,159.9bp) and (1226.6bp,179.46bp)  .. (1276.8bp,199.66bp);
  % Edge: SC_Compl_Consistent -> SC_Cut
  \draw [->,-stealth,very thin] (720.17bp,254.94bp) .. controls (747.03bp,271.19bp) and (781.47bp,292.04bp)  .. (817.72bp,313.98bp);
  % Edge: SC_Sema -> Sema
  \draw [->,-stealth,very thin] (712.86bp,355.68bp) .. controls (733.79bp,364.53bp) and (760.81bp,375.13bp)  .. (785.6bp,382.36bp) .. controls (856.53bp,403.04bp) and (940.17bp,418.07bp)  .. (1006.1bp,428.37bp);
  % Edge: ND_Compl_SC -> Compactness
  \draw [->,-stealth,very thin] (367.06bp,255.78bp) .. controls (393.56bp,274.04bp) and (428.04bp,297.82bp)  .. (461.37bp,320.81bp);
  % Edge: SCND -> SC
  \draw [->,-stealth,very thin] (189.01bp,355.95bp) .. controls (200.66bp,365.36bp) and (216.51bp,376.44bp)  .. (232.6bp,382.36bp) .. controls (310.18bp,410.87bp) and (536.97bp,425.53bp)  .. (678.45bp,432.29bp);
  % Edge: Resolution_Compl_SC_Full -> Resolution
  \draw [->,-stealth,very thin] (636.3bp,50.081bp) .. controls (658.42bp,54.112bp) and (681.81bp,58.343bp)  .. (703.6bp,62.225bp) .. controls (795.53bp,78.608bp) and (831.67bp,48.326bp)  .. (910.6bp,98.225bp) .. controls (939.23bp,116.32bp) and (927.82bp,139.72bp)  .. (954.6bp,160.45bp) .. controls (990.31bp,188.09bp) and (1007.9bp,179.19bp)  .. (1049.6bp,196.45bp) .. controls (1058.2bp,200.0bp) and (1067.3bp,203.96bp)  .. (1085.1bp,211.92bp);
  % Edge: MiniFormulas -> Formulas
  \draw [->,-stealth,very thin] (1253.5bp,451.98bp) .. controls (1214.1bp,463.41bp) and (1158.6bp,479.5bp)  .. (1107.2bp,494.42bp);
  % Edge: SC_Cut -> SC
  \draw [->,-stealth,very thin] (830.77bp,366.83bp) .. controls (817.59bp,380.77bp) and (801.69bp,397.58bp)  .. (781.76bp,418.66bp);
  % Edge: ND_Compl_Truthtable_Compact -> ND_Compl_Truthtable
  \draw [->,-stealth,very thin] (200.92bp,157.91bp) .. controls (186.71bp,168.6bp) and (170.29bp,180.94bp)  .. (147.18bp,198.32bp);
  % Edge: SC -> Formulas
  \draw [->,-stealth,very thin] (821.4bp,451.22bp) .. controls (870.85bp,463.16bp) and (943.02bp,480.6bp)  .. (1004.5bp,495.44bp);
  % Edge: Sema_Craig -> Substitution_Sema
  \draw [->,-stealth,very thin] (1744.4bp,253.83bp) .. controls (1709.9bp,272.74bp) and (1663.2bp,298.3bp)  .. (1621.7bp,321.01bp);
  % Edge: Resolution_Compl_SC_Small -> Resolution
  \draw [->,-stealth,very thin] (952.25bp,62.244bp) .. controls (955.24bp,90.454bp) and (963.56bp,132.36bp)  .. (986.6bp,160.45bp) .. controls (992.87bp,168.09bp) and (1039.8bp,190.98bp)  .. (1084.8bp,212.01bp);
  % Edge: LSC -> SC_Cut
  \draw [->,-stealth,very thin] (896.43bp,258.62bp) .. controls (890.18bp,270.99bp) and (882.87bp,285.47bp)  .. (871.54bp,307.93bp);
  % Edge: MiniFormulas_Sema -> Sema
  \draw [->,-stealth,very thin] (1319.2bp,369.14bp) .. controls (1307.5bp,373.91bp) and (1295.3bp,378.53bp)  .. (1283.6bp,382.36bp) .. controls (1228.5bp,400.45bp) and (1163.8bp,415.13bp)  .. (1108.0bp,426.52bp);
  % Edge: LSC -> CNF
  \draw [->,-stealth,very thin] (928.16bp,258.62bp) .. controls (937.38bp,275.37bp) and (948.73bp,295.99bp)  .. (962.37bp,320.75bp);
  % Edge: Resolution -> CNF
  \draw [->,-stealth,very thin] (1091.3bp,244.17bp) .. controls (1071.1bp,257.0bp) and (1042.3bp,276.03bp)  .. (1018.6bp,294.68bp) .. controls (1010.6bp,300.99bp) and (1002.2bp,308.34bp)  .. (987.19bp,322.25bp);
  % Edge: Resolution_Compl_SC_Full -> Compactness
  \draw [->,-stealth,very thin] (522.38bp,62.276bp) .. controls (517.14bp,73.296bp) and (511.85bp,86.048bp)  .. (508.6bp,98.225bp) .. controls (488.66bp,172.91bp) and (485.54bp,264.79bp)  .. (485.33bp,320.05bp);
  % Edge: HC_Compl_Consistency -> HC
  \draw [->,-stealth,very thin] (1951.2bp,257.37bp) .. controls (1910.8bp,298.44bp) and (1837.3bp,373.09bp)  .. (1792.4bp,418.67bp);
  % Edge: ND_Compl_SC -> SCND
  \draw [->,-stealth,very thin] (288.39bp,255.78bp) .. controls (261.58bp,274.38bp) and (226.53bp,298.71bp)  .. (193.31bp,321.76bp);
  % Edge: Sema -> Formulas
  \draw [->,-stealth,very thin] (1058.6bp,455.24bp) .. controls (1058.6bp,462.92bp) and (1058.6bp,472.13bp)  .. (1058.6bp,490.73bp);
  % Edge: Substitution_Sema -> Substitution
  \draw [->,-stealth,very thin] (1581.0bp,357.17bp) .. controls (1572.1bp,371.91bp) and (1559.3bp,393.13bp)  .. (1544.0bp,418.53bp);
  % Edge: HC_Compl_Consistency -> Consistency
  \draw [->,-stealth,very thin] (1919.6bp,250.03bp) .. controls (1908.1bp,253.45bp) and (1896.1bp,256.55bp)  .. (1884.6bp,258.68bp) .. controls (1621.9bp,307.41bp) and (1546.1bp,240.17bp)  .. (1284.6bp,294.68bp) .. controls (1268.5bp,298.04bp) and (1251.6bp,303.19bp)  .. (1226.1bp,312.3bp);
  % Edge: Substitution -> Formulas
  \draw [->,-stealth,very thin] (1474.4bp,446.56bp) .. controls (1388.9bp,459.31bp) and (1230.6bp,482.89bp)  .. (1125.8bp,498.5bp);
  % Edge: Resolution_Compl_SC_Full -> LSC_Resolution
  \draw [->,-stealth,very thin] (607.71bp,57.018bp) .. controls (653.05bp,73.361bp) and (711.43bp,94.4bp)  .. (762.67bp,112.86bp);
  % Edge: CNF_Sema -> Sema
  \draw [->,-stealth,very thin] (1236.8bp,249.51bp) .. controls (1155.9bp,265.89bp) and (1057.7bp,287.1bp)  .. (1051.6bp,294.68bp) .. controls (1025.9bp,326.86bp) and (1036.9bp,377.56bp)  .. (1051.1bp,418.33bp);
  % Edge: Resolution_Compl -> CNF_Sema
  \draw [->,-stealth,very thin] (1351.2bp,160.52bp) .. controls (1350.2bp,168.59bp) and (1349.2bp,177.45bp)  .. (1347.1bp,196.16bp);
  % Edge: Resolution_Compl_SC_Full -> SC_Sema
  \draw [->,-stealth,very thin] (531.14bp,62.473bp) .. controls (521.61bp,108.44bp) and (510.36bp,196.89bp)  .. (546.6bp,258.68bp) .. controls (562.85bp,286.38bp) and (593.01bp,305.55bp)  .. (629.66bp,322.2bp);
  % Edge: Compactness_Consistency -> Consistency
  \draw [->,-stealth,very thin] (1536.6bp,249.51bp) .. controls (1525.3bp,252.85bp) and (1513.7bp,256.06bp)  .. (1502.6bp,258.68bp) .. controls (1407.0bp,281.29bp) and (1379.6bp,269.68bp)  .. (1284.6bp,294.68bp) .. controls (1269.4bp,298.67bp) and (1253.4bp,303.85bp)  .. (1228.5bp,312.75bp);
  % Edge: ND_Sound -> Sema
  \draw [->,-stealth,very thin] (343.69bp,356.02bp) .. controls (359.98bp,365.45bp) and (381.75bp,376.54bp)  .. (402.6bp,382.36bp) .. controls (600.97bp,437.75bp) and (659.36bp,401.17bp)  .. (864.6bp,418.36bp) .. controls (907.26bp,421.93bp) and (955.17bp,426.22bp)  .. (1003.0bp,430.59bp);
  % Edge: LSC_Resolution -> LSC
  \draw [->,-stealth,very thin] (824.55bp,147.54bp) .. controls (837.65bp,159.43bp) and (855.57bp,175.7bp)  .. (879.51bp,197.43bp);
  % Edge: SC_Compl_Consistent -> Consistency
  \draw [->,-stealth,very thin] (739.36bp,249.57bp) .. controls (750.68bp,252.92bp) and (762.43bp,256.12bp)  .. (773.6bp,258.68bp) .. controls (878.74bp,282.74bp) and (908.11bp,272.18bp)  .. (1013.6bp,294.68bp) .. controls (1037.1bp,299.69bp) and (1062.3bp,306.22bp)  .. (1095.2bp,315.41bp);
  % Edge: SC_Sema -> SC
  \draw [->,-stealth,very thin] (692.51bp,356.72bp) .. controls (706.47bp,371.81bp) and (726.95bp,393.95bp)  .. (749.53bp,418.36bp);
  % Edge: ND_Compl_Truthtable -> ND_Sound
  \draw [->,-stealth,very thin] (160.48bp,255.22bp) .. controls (196.76bp,274.22bp) and (244.83bp,299.4bp)  .. (287.2bp,321.59bp);
  % Edge: Consistency -> Sema
  \draw [->,-stealth,very thin] (1134.9bp,368.39bp) .. controls (1119.0bp,382.4bp) and (1100.2bp,399.02bp)  .. (1077.3bp,419.2bp);
  % Edge: Resolution_Compl -> Resolution
  \draw [->,-stealth,very thin] (1287.7bp,155.54bp) .. controls (1256.3bp,167.59bp) and (1218.4bp,182.41bp)  .. (1184.6bp,196.45bp) .. controls (1175.9bp,200.05bp) and (1166.7bp,204.04bp)  .. (1148.6bp,212.01bp);
  % Edge: HC -> Formulas
  \draw [->,-stealth,very thin] (1703.4bp,445.55bp) .. controls (1674.8bp,448.6bp) and (1641.7bp,452.08bp)  .. (1611.6bp,455.13bp) .. controls (1446.1bp,471.88bp) and (1252.7bp,490.29bp)  .. (1133.8bp,501.48bp);
  % Edge: MiniFormulas_Sema -> MiniFormulas
  \draw [->,-stealth,very thin] (1350.5bp,379.49bp) .. controls (1341.5bp,390.14bp) and (1331.9bp,401.34bp)  .. (1317.3bp,418.53bp);
  % Edge: CNF_Sema -> CNF
  \draw [->,-stealth,very thin] (1235.5bp,249.13bp) .. controls (1144.8bp,266.6bp) and (1028.5bp,289.64bp)  .. (1018.6bp,294.68bp) .. controls (1009.0bp,299.55bp) and (999.9bp,306.93bp)  .. (985.25bp,321.32bp);
  % Edge: ND_Sound -> ND
  \draw [->,-stealth,very thin] (317.6bp,357.17bp) .. controls (317.6bp,371.41bp) and (317.6bp,391.7bp)  .. (317.6bp,418.2bp);
  % Edge: ND_Compl_SC -> ND_Sound
  \draw [->,-stealth,very thin] (324.82bp,258.91bp) .. controls (323.35bp,274.87bp) and (321.57bp,294.3bp)  .. (319.22bp,319.9bp);
  % Node: ND_Compl_SC
\begin{scope}
  \definecolor{strokecol}{rgb}{0.0,0.0,0.0};
  \pgfsetstrokecolor{strokecol}
  \draw (327.6bp,232.76bp) node {ND completeness};
  \draw (327.6bp,214.76bp) node {(via SC)};
\end{scope}
  % Node: Substitution_Sema
\begin{scope}
  \definecolor{strokecol}{rgb}{0.0,0.0,0.0};
  \pgfsetstrokecolor{strokecol}
  \draw (1591.6bp,338.52bp) node {Substitution lemma};
\end{scope}
  % Node: LSC_Resolution
\begin{scope}
  \definecolor{strokecol}{rgb}{0.0,0.0,0.0};
  \pgfsetstrokecolor{strokecol}
  \draw (805.6bp,129.34bp) node {LSC ↔ Resolution};
\end{scope}
  % Node: HC
\begin{scope}
  \definecolor{strokecol}{rgb}{0.0,0.0,0.0};
  \pgfsetstrokecolor{strokecol}
  \draw (1775.6bp,436.74bp) node {Hilbert Calculus};
\end{scope}
  % Node: SC_Compl_Consistent
\begin{scope}
  \definecolor{strokecol}{rgb}{0.0,0.0,0.0};
  \pgfsetstrokecolor{strokecol}
  \draw (676.6bp,232.76bp) node {SC complete};
  \draw (676.6bp,214.76bp) node {(via Consistency)};
\end{scope}
  % Node: SCND
\begin{scope}
  \definecolor{strokecol}{rgb}{0.0,0.0,0.0};
  \pgfsetstrokecolor{strokecol}
  \draw (170.6bp,338.52bp) node {SC→ND};
\end{scope}
  % Node: Formulas
\begin{scope}
  \definecolor{strokecol}{rgb}{0.0,0.0,0.0};
  \pgfsetstrokecolor{strokecol}
  \draw (1058.6bp,509.51bp) node {Formula Syntax};
\end{scope}
  % Node: LSC
\begin{scope}
  \definecolor{strokecol}{rgb}{0.0,0.0,0.0};
  \pgfsetstrokecolor{strokecol}
  \draw (911.6bp,232.76bp) node {Transforming SC: to CNF,};
  \draw (911.6bp,214.76bp) node {to left handed SC (LSC)};
\end{scope}
  % Node: ND
\begin{scope}
  \definecolor{strokecol}{rgb}{0.0,0.0,0.0};
  \pgfsetstrokecolor{strokecol}
  \draw (317.6bp,436.74bp) node {Natural Deduction};
\end{scope}
  % Node: MiniFormulas
\begin{scope}
  \definecolor{strokecol}{rgb}{0.0,0.0,0.0};
  \pgfsetstrokecolor{strokecol}
  \draw (1302.6bp,436.74bp) node {→ only formulas};
\end{scope}
  % Node: ND_Compl_Truthtable_Compact
\begin{scope}
  \definecolor{strokecol}{rgb}{0.0,0.0,0.0};
  \pgfsetstrokecolor{strokecol}
  \draw (237.6bp,134.54bp) node {ND completeness};
  \draw (237.6bp,116.54bp) node {(generalized)};
\end{scope}
  % Node: Substitution
\begin{scope}
  \definecolor{strokecol}{rgb}{0.0,0.0,0.0};
  \pgfsetstrokecolor{strokecol}
  \draw (1533.6bp,436.74bp) node {Substitutions};
\end{scope}
  % Node: ND_Sound
\begin{scope}
  \definecolor{strokecol}{rgb}{0.0,0.0,0.0};
  \pgfsetstrokecolor{strokecol}
  \draw (317.6bp,338.52bp) node {ND soundness};
\end{scope}
  % Node: Resolution_Compl_SC_Small
\begin{scope}
  \definecolor{strokecol}{rgb}{0.0,0.0,0.0};
  \pgfsetstrokecolor{strokecol}
  \draw (950.6bp,36.31bp) node {Resolution completeness via LSC};
  \draw (950.6bp,18.31bp) node {(using semantic correctness of formula CNF trans.)};
\end{scope}
  % Node: Sema_Craig
\begin{scope}
  \definecolor{strokecol}{rgb}{0.0,0.0,0.0};
  \pgfsetstrokecolor{strokecol}
  \draw (1790.6bp,232.76bp) node {Interpolation};
  \draw (1790.6bp,214.76bp) node {using Semantics};
\end{scope}
  % Node: MiniFormulas_Sema
\begin{scope}
  \definecolor{strokecol}{rgb}{0.0,0.0,0.0};
  \pgfsetstrokecolor{strokecol}
  \draw (1384.6bp,352.72bp) node {proof that};
  \draw (1384.6bp,334.72bp) node {→ transformation};
  \draw (1384.6bp,316.72bp) node {is sound};
\end{scope}
  % Node: ND_Compl_Truthtable
\begin{scope}
  \definecolor{strokecol}{rgb}{0.0,0.0,0.0};
  \pgfsetstrokecolor{strokecol}
  \draw (109.6bp,232.76bp) node {ND completeness};
  \draw (109.6bp,214.76bp) node {(simulates truthtables)};
\end{scope}
  % Node: Sema
\begin{scope}
  \definecolor{strokecol}{rgb}{0.0,0.0,0.0};
  \pgfsetstrokecolor{strokecol}
  \draw (1058.6bp,436.74bp) node {Semantics};
\end{scope}
  % Node: CNF
\begin{scope}
  \definecolor{strokecol}{rgb}{0.0,0.0,0.0};
  \pgfsetstrokecolor{strokecol}
  \draw (971.6bp,338.52bp) node {CNF};
\end{scope}
  % Node: SC
\begin{scope}
  \definecolor{strokecol}{rgb}{0.0,0.0,0.0};
  \pgfsetstrokecolor{strokecol}
  \draw (765.6bp,436.74bp) node {Sequent Calculus};
\end{scope}
  % Node: CNF_Sema
\begin{scope}
  \definecolor{strokecol}{rgb}{0.0,0.0,0.0};
  \pgfsetstrokecolor{strokecol}
  \draw (1343.6bp,232.76bp) node {Semantics of CNF formulas};
  \draw (1343.6bp,214.76bp) node {Correctness of CNF translation};
\end{scope}
  % Node: Compactness
\begin{scope}
  \definecolor{strokecol}{rgb}{0.0,0.0,0.0};
  \pgfsetstrokecolor{strokecol}
  \draw (485.6bp,338.52bp) node {Compactness};
\end{scope}
  % Node: Consistency
\begin{scope}
  \definecolor{strokecol}{rgb}{0.0,0.0,0.0};
  \pgfsetstrokecolor{strokecol}
  \draw (1167.6bp,343.72bp) node {Abstract Consistency};
  \draw (1167.6bp,325.72bp) node {Properties};
\end{scope}
  % Node: Resolution_Compl
\begin{scope}
  \definecolor{strokecol}{rgb}{0.0,0.0,0.0};
  \pgfsetstrokecolor{strokecol}
  \draw (1354.6bp,134.54bp) node {Resolution completeness};
  \draw (1354.6bp,116.54bp) node {(induction over atom set)};
\end{scope}
  % Node: Resolution_Sound
\begin{scope}
  \definecolor{strokecol}{rgb}{0.0,0.0,0.0};
  \pgfsetstrokecolor{strokecol}
  \draw (1104.6bp,129.34bp) node {Resolution soundness};
\end{scope}
  % Node: Resolution
\begin{scope}
  \definecolor{strokecol}{rgb}{0.0,0.0,0.0};
  \pgfsetstrokecolor{strokecol}
  \draw (1116.6bp,227.56bp) node {Resolution};
\end{scope}
  % Node: SC_Sema
\begin{scope}
  \definecolor{strokecol}{rgb}{0.0,0.0,0.0};
  \pgfsetstrokecolor{strokecol}
  \draw (676.6bp,338.52bp) node {SC sound/complete};
\end{scope}
  % Node: Resolution_Compl_SC_Full
\begin{scope}
  \definecolor{strokecol}{rgb}{0.0,0.0,0.0};
  \pgfsetstrokecolor{strokecol}
  \draw (538.6bp,36.31bp) node {Resolution completeness};
  \draw (538.6bp,18.31bp) node {via LSC};
\end{scope}
  % Node: HC_Compl_Consistency
\begin{scope}
  \definecolor{strokecol}{rgb}{0.0,0.0,0.0};
  \pgfsetstrokecolor{strokecol}
  \draw (1979.6bp,232.76bp) node {HC complete};
  \draw (1979.6bp,214.76bp) node {(via consistency)};
\end{scope}
  % Node: Compactness_Consistency
\begin{scope}
  \definecolor{strokecol}{rgb}{0.0,0.0,0.0};
  \pgfsetstrokecolor{strokecol}
  \draw (1599.6bp,232.76bp) node {Compactness};
  \draw (1599.6bp,214.76bp) node {(via Consistency)};
\end{scope}
  % Node: SC_Cut
\begin{scope}
  \definecolor{strokecol}{rgb}{0.0,0.0,0.0};
  \pgfsetstrokecolor{strokecol}
  \draw (856.6bp,343.72bp) node {SC: cut};
  \draw (856.6bp,325.72bp) node {contraction};
\end{scope}
%
\end{tikzpicture}


	\caption{Overview of results considering Semantics}
	\label{fig:sema}
\end{figure}

% sane default for proof documents
\parindent 0pt\parskip 0.5ex

% generated text of all theories
\input{session}

% optional bibliography
\bibliographystyle{abbrv}
\bibliography{root}

\end{document}

%%% Local Variables:
%%% mode: latex
%%% TeX-master: t
%%% End:
