\documentclass[11pt,a4paper]{article}
\usepackage[T1]{fontenc}
\usepackage{isabelle,isabellesym}

% further packages required for unusual symbols (see also
% isabellesym.sty), use only when needed

\usepackage{amssymb}
  %for \<leadsto>, \<box>, \<diamond>, \<sqsupset>, \<mho>, \<Join>,
  %\<lhd>, \<lesssim>, \<greatersim>, \<lessapprox>, \<greaterapprox>,
  %\<triangleq>, \<yen>, \<lozenge>

%\usepackage{eurosym}
  %for \<euro>

%\usepackage[only,bigsqcap,bigparallel,fatsemi,interleave,sslash]{stmaryrd}
  %for \<Sqinter>, \<Parallel>, \<Zsemi>, \<Parallel>, \<sslash>

%\usepackage{eufrak}
  %for \<AA> ... \<ZZ>, \<aa> ... \<zz> (also included in amssymb)

%\usepackage{textcomp}
  %for \<onequarter>, \<onehalf>, \<threequarters>, \<degree>, \<cent>,
  %\<currency>

% this should be the last package used
\usepackage{pdfsetup}

% urls in roman style, theory text in math-similar italics
\urlstyle{rm}
\isabellestyle{it}

% for uniform font size
%\renewcommand{\isastyle}{\isastyleminor}


\begin{document}

\title{Simple Clause Learning for First-Order Logic}
\author{Martin Desharnais}
\maketitle

\begin{abstract}
  This Isabelle/HOL formalization covers the unexecutable specification of Simple Clause Learning for first-order logic without equality \cite{Fiori2019,Bromberger2023}: SCL(FOL).
  The main results are formal proofs of soundness, non-redundancy of learned clauses, termination, and refutational completeness.
  Compared to the unformalized version, the formalized calculus is simpler, a number of results were generalized, and the non-redundancy statement was strengthened.
  We found one bug in a previously published version of the SCL Backtrack rule.
  Compared to related formalizations, we introduce a new technique for showing termination based on non-redundant clause learning.
\end{abstract}

\tableofcontents

% sane default for proof documents
\parindent 0pt\parskip 0.5ex

% generated text of all theories
\input{session}

% optional bibliography
\bibliographystyle{abbrv}
\bibliography{root}

\end{document}

%%% Local Variables:
%%% mode: latex
%%% TeX-master: t
%%% End:
