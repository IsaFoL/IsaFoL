%Some LaTeX checking: no bad pratices
%\RequirePackage[l2tabu, orthodox]{nag}
%\RequirePackage[all,error]{onlyamsmath}
\RequirePackage{fixltx2e}

\documentclass[11pt,a4paper]{book}
\usepackage{isabelle,isabellesym}

% further packages required for unusual symbols (see also
% isabellesym.sty), use only when needed

% lualatex
%\usepackage{spelling}
\usepackage{fullpage}
\usepackage{graphicx}
\usepackage{comment}


\usepackage{mdframed}
%% Saisie en UTF-8
\usepackage[utf8]{inputenc}
\usepackage[T1]{fontenc}
\usepackage{lmodern}
\usepackage{subcaption}

%% Pour composer des mathématiques
\usepackage{amsmath,amssymb, amsthm}
\usepackage{nicefrac}
\usepackage{tikz}
\usetikzlibrary{decorations, arrows, shapes, automata, mindmap, trees}
  %for \<leadsto>, \<box>, \<diamond>, \<sqsupset>, \<mho>, \<Join>,
  %\<lhd>, \<lesssim>, \<greatersim>, \<lessapprox>, \<greaterapprox>,
  %\<triangleq>, \<yen>, \<lozenge>

%\usepackage{eurosym}
  %for \<euro>

\usepackage[only,bigsqcap]{stmaryrd}
  %for \<Sqinter>
\usepackage{wasysym}

%\usepackage{eufrak}
  %for \<AA> ... \<ZZ>, \<aa> ... \<zz> (also included in amssymb)

%\usepackage{textcomp}
  %for \<onequarter>, \<onehalf>, \<threequarters>, \<degree>, \<cent>,
  %\<currency>

\usepackage[english]{babel}
% this should be the last package used
\usepackage{pdfsetup}

% urls in roman style, theory text in math-similar italics
\urlstyle{rm}
\isabellestyle{it}

\begin{document}


\title{Formalization of a comprehensive framework for saturation theorem proving in Isabelle/HOL}
\author{Sophie Tourret}
\maketitle
\tableofcontents

% sane default for proof documents
\parindent 0pt\parskip 0.5ex

This Isabelle/HOL formalization is the companion of the technical report ``A comprehensive framework for saturation theorem proving'', itself companion of the eponym IJCAR 2020 paper, written by Uwe Waldmann, Sophie Tourret, Simon Robillard and Jasmin Blanchette.
It verifies a framework for formal refutational completeness proofs of abstract provers that implement saturation calculi, such as ordered resolution or superposition and allows to model entire prover architectures in such a way that the static refutational completeness of a calculus immediately implies the dynamic refutational completeness of a prover implementing the calculus using a variant of the Given Clause loop.

The technical report ``A comprehensive framework for saturation theorem proving'' is available at \url{http://matryoshka.gforge.inria.fr/pubs/satur\_report.pdf} .
The names of the Isabelle lemmas and theorems corresponding to the results in the report are indicated in the margin of the report.

% generated text of all theories
\input{session}

% optional bibliography
%\bibliographystyle{abbrv}
%\bibliography{root}

\end{document}

%%% Local Variables:
%%% mode: latex
%%% TeX-master: t
%%% End:
