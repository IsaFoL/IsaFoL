\documentclass[fleqn]{beamer}


\mode<presentation>
{
  \usetheme{default}
%   \useinnertheme[shadow=true]{rounded}

  \useinnertheme{circles}
  
%    \useoutertheme{infolines}
  % or ...
  \setbeamersize{text margin left=1em,text margin right=1em}

%   \setbeamercovered{transparent}
  % or whatever (possibly just delete it)

  \beamertemplatenavigationsymbolsempty
  
% Display frame numbers in footline
  \setbeamertemplate{footline}[frame number]
}

\usepackage{etex}
\usepackage[utf8]{inputenc}
\usepackage[T1]{fontenc}
\usepackage[ngerman]{babel}
\usepackage{amsmath}
% \usepackage{amsthm}
% \usepackage{stmaryrd}
\usepackage{times}
\usepackage{mathpartir}

\usepackage{centernot}

\usepackage{colortbl}
\usepackage{multirow}

\usepackage[purexy]{qsymbols}
\usepackage{graphicx}

\usepackage{listings}
\usepackage{lstautogobble}
\usepackage{relsize}

\usepackage{ifthen}
\usepackage{pgf}
\usepackage{tikz}
\usetikzlibrary{scopes}
\usetikzlibrary{decorations}
  \usetikzlibrary{decorations.pathmorphing}
\usetikzlibrary{arrows}
\usetikzlibrary{automata}
\usetikzlibrary{positioning}
\usetikzlibrary{chains}
\usetikzlibrary{shapes.geometric}
\usetikzlibrary{shapes.callouts}
\usetikzlibrary{shapes.misc}
\usetikzlibrary{shapes.multipart}
\usetikzlibrary{fit}
\usetikzlibrary{calc}
\usepackage{pgfplots}

% \tikzstyle{stack}=[inner sep=0pt,minimum size=2mm]
% \tikzstyle{ssline}=[->,snake=snake,segment amplitude=.2mm,segment length=3mm,line after snake=1mm]
% \tikzstyle{fgnode}=[circle,draw,inner sep=0pt,minimum size=2mm]


\usepackage{packages/isabelle}
\usepackage{packages/isabelletags}
\usepackage{packages/isabellesym}
\usepackage{packages/comment}

% \isabellestyle{it}

\def\isachardoublequote{}%
\def\isachardoublequoteopen{}%
\def\isachardoublequoteclose{}%

\newcommand{\isainnerkeyword}[1]{{\textbf{#1}}}
\newcommand{\isasymexistsA}{\isamath{\exists_{\textsc A}\,}}


\def\isadelimproof{}
\def\endisadelimproof{}
\def\isatagproof{}
\def\endisatagproof{}
\def\isafoldproof{}
\def\isadelimproof{}
\def\endisadelimproof{}

\def\isastylescript{\sl}%


% Not really meant for highlighting isabelle source, but for easily writing latex that looks like
% isabelle
% 
% keyword level 1 - isabelle outer syntax 
% keyword level 2 - isabelle inner syntax programming constructs (if, let, etc)
% keyword level 3 - standard constants (length, mod, etc)
% keyword level 4 - isabelle proof methods

\definecolor{isa-highlight-1}{rgb}{.6,0,0}

\lstdefinelanguage{isabelle}{
  morekeywords={theorem,theorems,corollary,lemma,lemmas,locale,begin,end,fixes,assumes,shows,and,
    constrains , definition, where, apply, done,unfolding, primrec, fun, using, by, for, uses,
    schematic_lemma, concrete_definition, prepare_code_thms, export_code, datatype, type_synonym,
    proof, next, qed, show, have, hence, thus, interpretation, fix, context, sepref_definition,is
 } ,
  morekeywords=[2]{rec, return, bind, foreach, if, then, else, do, let, in, res, spec, fail, assert, while, case, of,
    check},
%  morekeywords=[3]{length,mod,insert},
%   morekeywords=[4]{simp,auto,intro,elim,rprems,refine_mono,refine_rcg},
  sensitive=True,
  morecomment=[s]{(\*}{\*)},
  moredelim=**[is][\color{isa-highlight-1}]{/@}{@/}
}

\lstset{
    language=isabelle,
    mathescape=true,
    escapeinside={--"}{"},
    basicstyle={\itshape},
    keywordstyle=\rm\bfseries,
    keywordstyle=[2]\rm\tt,
    keywordstyle=[3]\rm,
    keywordstyle=[4]\rm,
    showstringspaces=false,
    keepspaces=true,
    columns=[c]fullflexible}
\lstset{literate=
  {"}{}0
  {'}{{${}^\prime$}}1
  {\%}{{$\lambda$}}1
  {\\\%}{{$\lambda$}}1
  {\\\$}{{$\mathbin{\,\$\,}$}}1
  {->}{{$\rightarrow$}}1
  {<-}{{$\leftarrow$}}1
  {<.}{{$\langle$}}1
  {.>}{{$\rangle$}}1
  {<=}{{$\le$}}1
  {<->}{{$\leftrightarrow$}}1
  {-->}{{$\longrightarrow$}}2
  {<-->}{{$\longleftrightarrow$}}1
  {=>}{{$\Rightarrow$}}1
  {==}{{$\equiv$}}2
  {==>}{{$\implies$}}2
  {<=>}{{$\Leftrightarrow$}}1
  {~=}{{$\ne$}}1
  {-`}{{$\rightharpoonup$}}1
  {|`}{{$\restriction$}}1
  {!!}{{$\bigwedge$}}1
  {(}{{$($}}1
  {)}{{$)$}}1
  {\{}{{$\{$}}1
  {\}}{{$\}$}}1
  {[}{{$[$}}1
  {]}{{$]$}}1
  {(|}{{$\lrec$}}1
  {|)}{{$\rrec$}}1
  {[|}{{$\lsem$}}1
  {|]}{{$\rsem$}}1
  {\\<lbrakk>}{{$\lsem$}}1
  {\\<rbrakk>}{{$\rsem$}}1
  {|-}{{$\vdash$}}1
  {|->}{{$\mapsto$}}1
  {|_|}{{$\bigsqcup$}}1
  {...}{{$\dots$}}1
  {\\x}{{$\times$}}1
  {_0}{{${}_0$}}1
  {_1}{{${}_1$}}1
  {_2}{{${}_2$}}1
  {_3}{{${}_3$}}1
  {_4}{{${}_4$}}1
  {_5}{{${}_5$}}1
  {_6}{{${}_6$}}1
  {_7}{{${}_7$}}1
  {_8}{{${}_8$}}1
  {_9}{{${}_9$}}1
  {^*}{{$^*$}}1
  {\\<^sup>*}{{$^*$}}1
  {\\<^sub>*}{{$_*$}}1
  {\\<^sub>A}{{$_A$}}1
  {\\<^sub>r}{{$_r$}}1
  {\\<^sub>a}{{$_a$}}1
  {:_i}{{$:_i$}}1
  {\\<A>}{{$\mathcal{A}$}}1
  {\\<O>}{{\sf o}}1
  {\\<Phi>}{{$\Phi$}}1
  {\\<Psi>}{{$\Psi$}}1
  {\\<sigma>}{{$\sigma$}}1
  {\\<in>}{{$\in$}}1
  {\\<le>}{{$\le$}}1
  {\\<noteq>}{{$\ne$}}1
  {\\<lambda>}{{$\lambda$}}1
  {\\<longrightarrow>}{{$\longrightarrow$}}1
  {\\<longleftrightarrow>}{{$\longleftrightarrow$}}1
  {\\<Rightarrow>}{{$\Rightarrow$}}1
  {\\<Longrightarrow>}{{$\Longrightarrow$}}1
  {\\<rightarrow>}{{$\rightarrow$}}1
  {\\<leftarrow>}{{$\leftarrow$}}1
  {\\<mapsto>}{{$\mapsto$}}1
  {\\<equiv>}{{$\equiv$}}1
  {\\<and>}{{$\wedge$}}1
  {\\<or>}{{$\vee$}}1
  {\\<And>}{{$\bigwedge$}}1
  {\\<Up>}{{$\Uparrow$}}1
  {\\<Down>}{{$\Downarrow$}}1
  {\\<up>}{{$\uparrow$}}1
  {\\<down>}{{$\downarrow$}}1
  {\\<times>}{{$\times$}}1
  {\\<forall>}{{$\forall$}}1
  {\\<exists>}{{$\exists$}}1
  {\\<nexists>}{{$\nexists$}}1
  {\\<union>}{{$\cup$}}1
  {\\<Union>}{{$\bigcup$}}1
  {\\<inter>}{{$\cap$}}1
  {\\<subset>}{{$\subset$}}1
  {\\<subseteq>}{{$\subseteq$}}1
  {\\<supset>}{{$\supset$}}1
  {\\<supseteq>}{{$\supseteq$}}1
  {\\<alpha>}{{$\alpha$}}1
  {\\<beta>}{{$\beta$}}1
  {\\<gamma>}{{$\gamma$}}1
  {\\alpha}{{$\alpha$}}1
  {\\beta}{{$\beta$}}1
  {\\gamma}{{$\gamma$}}1
  {\\<Gamma>}{{$\Gamma$}}1
  {\\<langle>}{{$\langle$}}1
  {\\<rangle>}{{$\rangle$}}1
  {\\<not>}{{$\neg$}}1
  {\\<notin>}{{$\notin$}}1
  {\\<guillemotright>}{{$\gg$}}1
  {\\in}{$\in$}1
  {\\and}{$\wedge$}1
  {\\or}{$\vee$}1
  {\\Phi}{{$\Phi$}}1
  {\\Psi}{{$\Psi$}}1
  {\\le}{{$\le$}}1
  {\\Up}{{$\Uparrow$}}1
  {\\Down}{{$\Down$}}1
  {>>}{{$\gg$}}1
  {>>=}{{${\gg}{=}$}}1
  {<*lex*>}{{$\times_{\sf lex}$}}1
}


\newcommand{\isai}{\lstinline[language=isabelle,basicstyle=\normalsize\ttfamily\slshape]}

\newcommand\CC{C\nolinebreak[4]\hspace{-.05em}\raisebox{.3ex}{\relsize{-1}{\textbf{++}}}}

\newcommand{\eqdef}{\mathrel{{=}_{def}}}
\newcommand{\iffdef}{\mathrel{{\mathord{\iff}\!\!}_{def}}}


\makeatletter
\newcommand*{\overlaynumber}{\number\beamer@slideinframe}
\makeatother

\AtBeginSection[] % Do nothing for \section*
{
  \begin{frame}<beamer>
    \frametitle{Outline}
    \tableofcontents[currentsection]
  \end{frame}
}


\title{The GRAT Tool Chain}
\subtitle{Efficient (UN)SAT Certificate Checking with Formal Correctness Guarantees}

% \subtitle
% {Subtitle} % (optional)

\author[Peter Lammich]{\underline{Peter Lammich}}
% - Use the \inst{?} command only if the authors have different
%   affiliation.

\institute[TUM/VT] % (optional, but mostly needed)
{ TU M\"unchen / Virginia Tech %, Institut f\"ur Informatik, Chair for Logic and Verification
}
% - Use the \inst command only if there are several affiliations.
% - Keep it simple, no one is interested in your street address.

\date {September 2017}
% {2008-12-01}


% If you have a file called "university-logo-filename.xxx", where xxx
% is a graphic format that can be processed by latex or pdflatex,
% resp., then you can add a logo as follows:

% \pgfdeclareimage[height=0.5cm]{university-logo}{university-logo-filename}
% \logo{\pgfuseimage{university-logo}}


% Delete this, if you do not want the table of contents to pop up at
% the beginning of each subsection:


% If you wish to uncover everything in a step-wise fashion, uncomment
% the following command: 

%\beamerdefaultoverlayspecification{<+->}

%\mathchardef\-="2D
%\renewcommand\-{\text{-}}

\newcommand{\mc}{\color{blue}}
\newcommand{\term}[1]{{\mc#1}}

\let\olddisplaystyle\displaystyle
\newcommand{\mydisplaystyle}{\olddisplaystyle\mc}
\let\displaystyle\mydisplaystyle

\newcommand{\smc}{\everymath{\mc}}
\smc

\lstset{basicstyle=\color{blue}}

% \newcommand<>{\btikzset}[2]{\alt#3{\tikzset{#1}}{\tikzset{#2}}}

\tikzset{onslide/.code args={<#1>#2}{%
  \only<#1>{\pgfkeysalso{#2}} % \pgfkeysalso doesn't change the path
}}

\tikzset{>=latex}


\lstset{autogobble}

\newcommand{\natN}{{\text{nat}_{\mathord{<}N}}}


\begin{document}
% % General
\newcommand{\ie}{i.\,e.\ }
\newcommand{\eg}{e.\,g.\ }
\newcommand{\wrt}{wrt.\ }
\newcommand{\cf}{cf.\ }
\newcommand{\etc}{etc.\ }


%Isabelle/HOL
\newcommand{\SOME}{\varepsilon}
\newcommand{\union}{\cup}
\newcommand{\inter}{\cap}
\renewcommand{\and}{\land}

\newcommand{\set}{{\sf set}}

\newcommand{\Sup}{\bigsqcup}
\newcommand{\sle}{\sqsubseteq}

\newcommand{\la}{{\sf \gets}}

\newcommand{\True}{{\sf True}}
\newcommand{\False}{{\sf False}}
\newcommand{\nat}{{\mathbb N}}
\newcommand{\bool}{{\mathbb B}}
\newcommand{\unit}{{\sf unit}}
\newcommand{\last}{{\sf last}}
\newcommand{\butlast}{{\sf butlast}}

\newcommand{\V}{V}

\renewcommand{\implies}{\Longrightarrow}

\newcommand{\isaconst}[1]{\textsl{#1}}
\newcommand{\isakeyw}[1]{\textsf{\bf #1}}
\newcommand{\isalem}[1]{\textsf{#1}}

\newcommand{\locale}{\isakeyw{locale}}

\newcommand{\lrec}{\mathopen{(\!|}}
\newcommand{\rrec}{\mathclose{|\!)}}

\newcommand{\lsem}{\ensuremath{\mathopen{[\![}}}
\newcommand{\rsem}{\ensuremath{\mathclose{]\!]}}}

\newcommand{\up}{\uparrow}

% Refinement Framework
\newcommand{\SUCCEED}{{\sf\bf succeed}}
\newcommand{\FAIL}{{\sf\bf fail}}
\newcommand{\fail}{{\sf\bf fail}}
\newcommand{\RES}{{\sf\bf res}}
\newcommand{\SPEC}{{\sf\bf spec}}
\newcommand{\return}{\RETURN}
\newcommand{\bind}{\BIND}
\newcommand{\RETURN}{{\sf\bf return}}
\newcommand{\BIND}{{\sf\bf bind}}
\newcommand{\assert}{\ASSERT}
\newcommand{\ASSERT}{{\sf\bf assert}}
\renewcommand{\do}{\DO}
\newcommand{\DO}{{\sf\bf do}}
\newcommand{\IN}{{\sf\bf in}}
\newcommand{\IF}{{\sf\bf if}}
\newcommand{\THEN}{{\sf\bf then}}
\newcommand{\ELSE}{{\sf\bf else}}
\newcommand{\LET}{{\sf\bf let}}
\newcommand{\WHILE}{{\sf\bf while}}
\newcommand{\WHILET}{\WHILE_{\sf\bf T}}
\newcommand{\FOREACH}{{\sf\bf foreach}}
\newcommand{\FOREACHC}{\FOREACH_{\sf\bf C}}
\newcommand{\REC}{{\sf\bf rec}}
\newcommand{\RECT}{{\sf\bf rec_T}}

\newcommand{\Up}{{\Uparrow}}
\newcommand{\Down}{{\Downarrow}}

\newcommand{\lfp}{{\sf lfp}}
\newcommand{\gfp}{{\sf gfp}}
\newcommand{\body}{{\sf B}}
\newcommand{\htrip}[3]{{\models\{#1\}~{#2}~\{#3\}}}

\newcommand{\mono}{{\sf mono}}
\newcommand{\wf}{{\sf wf}}

%LTL
\newcommand{\Prop}{\mathsf{Prop}}
\newcommand{\Next}{\mathord{\mathsf{X}}}
\newcommand{\Finally}{\mathord{\mathsf{F}}}
\newcommand{\Globally}{\mathord{\mathsf{G}}}
\renewcommand{\Until}{\mathbin{\mathsf{U}}}
\newcommand{\Release}{\mathbin{\mathsf{V}}}

%Automata
\newcommand{\GBA}{\mathord{\mathsf{GBA}}}
\newcommand{\LGBA}{\mathord{\mathsf{LGBA}}}

\newcommand{\QGBA}{{{Q\_GBA}}}
\newcommand{\DGBA}{{{\Delta\_GBA}}}
\newcommand{\IGBA}{{{I\_GBA}}}
\newcommand{\FGBA}{{{F\_GBA}}}
\newcommand{\LLGBA}{{{L}}}

\newcommand{\BAtrans}{\mathsf{trans}}
\newcommand{\BAinitial}{\mathsf{initial}}
\newcommand{\BAaccept}{\mathsf{accept}}

%Misc
\newcommand{\equivalent}{\leftrightarrow}

\newcommand{\expand}{\mathbin{\mathsf{expand}}}
\newcommand{\expandbody}{\mathbin{\mathsf{expand\_body}}}
\def\by#1{\mathop{{\hbox{\setbox0=\hbox{$\scriptstyle{#1\quad}$}{$\buildrel{\>#1\>}\over{\hbox to \wd0{\rightarrowfill}}$}}}}}

%personalized comments
\newcommand{\as}[1]{\textcolor{blue}{\sf AS: #1}}
\newcommand{\AS}[1]{\marginpar{\footnotesize\textcolor{blue}{\sf AS: #1}}}
\newcommand{\RN}[1]{\marginpar{\footnotesize\textcolor{green}{\sf RN: #1}}}
\newcommand{\JE}[1]{\marginpar{\footnotesize\textcolor{green}{\sf JE: #1}}}
\newcommand{\PL}[1]{\marginpar{\footnotesize\textcolor{red}{\sf PL: #1}}}
\newcommand{\jgs}[1]{\textcolor{magenta}{\sf jgs: #1}}
\newcommand{\JGS}[1]{\marginpar{\footnotesize\textcolor{magenta}{\sf JGS: #1}}}



\begin{frame}
  \titlepage
\end{frame}


# Motivation
  *<+-> SAT solvers
    * Elaborate optimizations + heuristics
    * Highly complex $\implies$ error prone
    * Used in safety critical applications
  *<+-> Certification
    * SAT: Solution. Easy!
    * UNSAT: De-facto standard: DRAT
  *<+-> DRAT certificate
    * Still requires highly optimized checker
    * Main work spent on unit propagation (BCP)
    * drat-trim is optimized C program
      * We found several bugs (now fixed)
    
# DRAT
  * Certificate is list of clause additions and deletions
  * Added clauses proved redundant
    * Must have RUP or RAT property
    * Proof based on BCP
  * Last clause: Empty clause
  
# Enriched Certificates
  Crucial idea 
    * Unverified \emph{generator} records unit/conflict clauses
    * Verified \emph{checker} replays BCP
    * \emph{Finding} unit clause replaced by \emph{checking} unit clause
  \pause  
  
  History
    * Cruz-Filipe et al.~[TACAS 2017] for DRUP
    * P.~L. [CADE 2017] extended to DRAT
    * independently: Cruz-Filipe et al.~[CADE 2017] extended to DRAT
  
  
# Workflow
  \begin{tikzpicture}[node distance = 1.5cm, auto]
    \tikzstyle{tool}=[shape=rectangle,draw, minimum size = 5mm]
    \tikzstyle{artf}=[minimum size = 5mm]
    
    \node[tool] (sats) {SAT Solver};
    \uncover<2->{\node[artf] (drat) [below of = sats] {DRAT certificate};}
    \uncover<3->{\node[tool] (gratgen) [below of = drat] {gratgen};}
    \uncover<1->{\node[artf] (cnf) [left of = gratgen, anchor=east] {CNF formula};}
    \uncover<4->{\node[artf] (grat) [below of = gratgen] {GRAT certificate};}
    \uncover<5->{\node[tool] (gratchk) [below of = grat] {gratchk};}
    \uncover<6->{\node[artf] (res) [right of = gratchk, anchor=west] {s UNSAT / s ERROR};}
    
  
    \draw[->] (cnf) |- (sats);
    \uncover<2->{\draw[->] (sats) to (drat);}
    \uncover<3->{\draw[->] (cnf) to (gratgen);}
    \uncover<3->{\draw[->] (drat) to (gratgen);}
    \uncover<4->{\draw[->] (gratgen) to (grat);}
    \uncover<5->{\draw[->] (grat) to (gratchk);}
    \uncover<6->{\draw[->] (gratchk) to (res);}
    \uncover<5->{\draw[->] (cnf) |- (gratchk);}
    
    
    \node<7->[rectangle callout, draw, callout absolute pointer=(gratgen.east), above right = of gratgen, align=left] {Expensive \\ Complex \\ Highly optimized};
    \node<8->[rectangle callout, draw, callout absolute pointer=(gratchk.east), above right = of gratchk, align=left] {Cheap \\ Simple \\ Formally verified};
    
  \end{tikzpicture}  
  
# Backwards Checking
  * Forward phase: Add lemmas, but do no proofs
  * Backwards phase: Remove lemma and prove it
    * Mark lemmas required for proof
    * Skip unmarked lemmas
  * Core first heuristics
    * Prefer marked lemmas when searching for proof
    * Goal: Reduce newly marked lemmas
  
# Novel Optimizations (Compared to drat-trim)
  * Multithreaded proving
    * Multiple threads in backwards phase
    * Synchronize on proved/marked lemmas
    * Significantly decreases wall-clock time
    * Significantly increases memory usage
  * Separate watchlists
    * Maintain separate watchlists for marked/unmarked clauses
    * Less work in (hot) unit propagation loop
    * More work in (cool) analyzer that marks clauses
  * RAT-run heuristics
    * Re-use gathered RAT candidates for next lemma's proof
    * RAT candidates are particular expensive to gather
    * Reuse possible in many cases

% # Perhaps animation:
%   * One thread: Goto next marked lemma, unit-propagation, analyze conflict, mark lemmas
%   
%   * Multiple threads: Goto next marked lemma, try to acquire, up, ana-conf, mark
    
    
\newcommand{\annot}[1]{{\footnotesize\color{red}{#1}}}

\newcommand{\uc}[2]{\uncover<#1->{#2}}  
# Benchmark Results
  \begin{uncoverenv}<1->
  Using 2016 SAT competition main track:
    * cmsat: 110 UNSAT, 64 SAT  (many RAT lemmas)
    * riss6: 128 UNSAT          (Silver medalist, but only RUP) 
  \end{uncoverenv}  

  \vfill
    
    
  \begin{uncoverenv}<2->
  \begin{tabular}{c|c|c|c|c|}
            & lrat*                             & drat-trim                         & grat/1          & grat/8                    \\\hline
      cmsat & \uc{3}{51h \annot{2xT/O, 1xSEGF}} & \uc{4}{42h \annot{2xT/O, 1xSEGF}} & \uc{5}{17h}     & \uc{6}{7h               } \\\hline
      riss6 & \uc{3}{42h}                       & \uc{4}{30h                      } & \uc{5}{26h}     & \uc{6}{14h \annot{1xOOM}} \\\hline\hline
    $\Sigma$& \uc{3}{93h}                       & \uc{4}{72h                      } & \uc{5}{44h}     & \uc{6}{21h              } \\\hline
  \end{tabular}
  {\tiny\vspace*{1em}}
  
  {\tiny Wall-clock times, excluded the 3 certificates that drat-trim failed on\\
  \begin{tabular}{ll}
  {\color{red} T/O} &--- Timeout after 20.000s \\
  {\color{red} SEGF} &--- crashed with SIGSEGV \\
  {\color{red} OOM} &--- out of memory (128GiB) \\
   {*} &--- Incremental version, [Heule et al., ITP'17] \\
  \end{tabular}
  }
  \end{uncoverenv}  
  
  \vfill
  \begin{uncoverenv}<7->
  gratchk time: 12\% (cmsat: 6\%, riss6: 16\%)
  \end{uncoverenv}  
  
  \vfill
  
  \begin{uncoverenv}<8->
  SAT mode: gratchk verified all 64 certificates in 40s
  \end{uncoverenv}  
  

#![t] Demo
%   * Gratgen, gratchk, gratgen/8
%   * If possible, on lxcisa0!
% 
%   * Perhaps gratgen no-core-first?
%   
%   * Use rotmul.miter.shuffled-as.sat03-356 ... good to demonstrate RAT-run heuristics, multithreading speedup
  \scriptsize
  \begin{onlyenv}<1>
  \begin{verbatim}  
  ls -1hs rotmul.miter.shuffled-as.sat03-356.{cnf,drat}
  688K rotmul.miter.shuffled-as.sat03-356.cnf
   30M rotmul.miter.shuffled-as.sat03-356.drat
  \end{verbatim}
  \end{onlyenv}

  \begin{onlyenv}<2>
  \begin{verbatim}  
  $ drat-trim rotmul.miter.shuffled-as.sat03-356.{cnf,drat}
  c parsing input formula with 5980 variables and 35229 clauses
  c finished parsing
  c detected empty clause; start verification via backward checking
  c 30476 of 35229 clauses in core
  c 84401 of 219768 lemmas in core using 3834963 resolution steps
  c 2162 RAT lemmas in core; 37038 redundant literals in core lemmas
  s VERIFIED
  c verification time: 13.773 seconds
  \end{verbatim}  
  \end{onlyenv}

  \begin{onlyenv}<3>
  \begin{verbatim}  
  $ time ./grat.sh rotmul.miter.shuffled-as.sat03-356
  [... output shortened ...]
  c Timing statistics (ms)
  c Overall:  7246

  c Lemma statistics
  c RUP lemmas:  79298
  c RAT lemmas:  2183
  c   RAT run heuristics:   1292

  c Reading cnf
  c Reading lemmas
  c Done
  c Verifying unsat (split certificate)
  s VERIFIED UNSAT

  real    0m1.012s
  user    0m0.984s
  sys     0m0.028s

  real    0m8.268s
  user    0m8.168s
  sys     0m0.100s

  \end{verbatim}  
  \end{onlyenv}

  \begin{onlyenv}<4>
  \begin{verbatim}  
  $ time ./grat.sh rotmul.miter.shuffled-as.sat03-356 -j 3
  [... output shortened ...]
  c Timing statistics (ms)
  c Overall:  4386

  c Lemma statistics
  c RUP lemmas:  79529
  c RAT lemmas:  2169
  c   RAT run heuristics:   1269

  c Reading cnf
  c Reading lemmas
  c Done
  c Verifying unsat (split certificate)
  s VERIFIED UNSAT

  real    0m1.011s
  user    0m0.988s
  sys     0m0.020s

  real    0m5.407s
  user    0m12.120s
  sys     0m0.124s
  
  \end{verbatim}  
  \end{onlyenv}

  \begin{onlyenv}<5>
  \begin{verbatim}  
  $ ls -1hs rotmul.miter.shuffled-as.sat03-356.grat?
  5,5M rotmul.miter.shuffled-as.sat03-356.gratl
   17M rotmul.miter.shuffled-as.sat03-356.gratp
  \end{verbatim}  
  \end{onlyenv}

  
  \begin{onlyenv}<6>
  \begin{verbatim}  
  $ ls -1hs unif-c500-v250-s1228594393.{cnf,sat}
  8,0K unif-c500-v250-s1228594393.cnf
  4,0K unif-c500-v250-s1228594393.sat
  
  $ time gratchk sat unif-c500-v250-s1228594393.{cnf,sat}
  c Reading cnf
  c Reading proof
  c Done
  c Verifying sat
  s VERIFIED SAT

  real    0m0.001s
  user    0m0.000s
  sys     0m0.000s
  \end{verbatim}  
  \end{onlyenv}
  
  
# Conclusions
  Fast SAT solver certification tool
  * SAT and UNSAT certificates
  * Strong formal correctness guarantees
    * formally verified down to int array representing formula
  * Available under BSD-style license
    
  \vfill 
  \center \url{http://www21.in.tum.de/~lammich/grat/}
  
  
%   Plot:
%     *# Use rotmul.miter.shuffled-as.sat03-356
%     * run drat-trim
%     * run grat toolchain
%     * run grat toolchain with multiple threads (laptop 3, lxcisa0 8)
%     * Show sizes of certificates - effective trimming!
    
  
  
  
  
  
  

% % ORIGINAL CADE TALK
% 
% # SAT
% %   *<+-> Find solution to cnf formula
% %     * Widely used, e.g. in hardware verification
% %     * NP-complete (intractable)
%   *<+-> SAT solvers
%     * Elaborate optimizations + heuristics
%     * Highly complex $\implies$ error prone
%     * Safety critical
%   *<+-> Certification
%     * SAT solver produces certificate
%     * Checked by simpler program
%   *<+-> SAT certificate
%     * Valuation, almost trivial to check
%   *<+-> UNSAT certificate
%     * De-facto standard: DRAT
%     * Demanded by SAT competitions
%     * Supported by most modern SAT solvers
%     
% # DRAT-Certificate
%   *<+-> List of clause additions (lemmas) and deletions
%   *<+-> Each lemma is \emph{redundant} wrt. current clauses
%     *<+-> Must have RAT property
%   *<+-> Last lemma is empty clause
%     
%   \uncover<+->{  
%   \begin{definition}
%     Clause $C$ has RAT wrt.\ formula $F$ iff 
%       *{}: $(F \wedge \neg C)^{\textrm u} = \{\square\}$
%       *or: $\exists l\in C.~\forall D\in F.~\neg l\in D \implies ((F\wedge \neg C)^{\textrm u}\wedge\neg D)^{\textrm u} = \{\square\}$
%     where $F^{\textrm u}$ means \emph{unit propagation}.
%   \end{definition}
%   }
%     
%     
% # DRAT in Practice
%   *<+-> Unit propagation and finding RAT candidates
%     * Computationally expensive
%   *<+-> Requires highly optimized code and data structures
%     * Error prone + hard to (formally) verify
%   *<+-> drat-trim (Heule et al.) 
%     * Standard tool used in SAT competitions
%     * Written in C, highly optimized, and many features
%     * Two-watched-literals, backwards checking, core-first, ...
%     * Sat/Unsat, trimming, unsat core, resolution graph, ...
% %     * Optimizations: 2wl, bwd checking, core-first, ...
% %     * Features: Sat/Unsat, trimming, unsat core, resolution graph, ...
%     *<+-> We found a few bugs (now mostly fixed)
%       * Erroneous pivot literal check
%       * Numeric overflows in parser
%       * Buffer overflows
%       
% 
% # Story of GRAT      
%   *<+-> I was verifying fast DRAT-checker (in Isabelle)
%   *<+-> Heard about GRIT [Cruz-Filipe et al., TACAS'17]
%     * Two-phase checker, but only for DRUP
%     * First phase: adapted drat-trim tool
%     * Second phase: verified in Coq
%     * Both phases: Approx. 2x slower than drat-trim
%   *<+-> Re-targeted my work to extend this to full DRAT
%     * Adding multi-threading to first phase
%     * Using my powerful refinement tools for fast second phase
%     * My tool was on par with drat-trim, albeit formally verified
%   *<+-> After CADE-deadline
%     * Added novel optimizations
%     * Now significantly faster than drat-trim
%     * New results included in this talk + proceedings paper
%   *<+-> Independently: LRAT was developed
%     * Cruz-Filipe, Heule, et al.
%       
%     
% # Enriched Certificates
%   \begin{onlyenv}<1>
%   \begin{tikzpicture}[node distance = 1.5cm, auto]
%     \tikzstyle{tool}=[shape=rectangle,draw, minimum size = 5mm]
%     \tikzstyle{artf}=[minimum size = 5mm]
%     
%     \node[artf] (drat) {DRAT certificate};
%     \node[tool] (drat-trim) [below of = drat] {drat-trim};
%     \node[artf] (cnf) [left of = drat-trim, anchor=east] {CNF formula};
%     \node[artf] (res) [right of = drat-trim, anchor=west] {s UNSAT / s ERROR};
%   
%     \draw[->] (cnf) to (drat-trim);
%     \draw[->] (drat) to (drat-trim);
%     \draw[->] (drat-trim) to (res);
%   \end{tikzpicture}  
%   \end{onlyenv}
%   
%   \begin{onlyenv}<2->
%   \begin{tikzpicture}[node distance = 1.5cm, auto]
%     \tikzstyle{tool}=[shape=rectangle,draw, minimum size = 5mm]
%     \tikzstyle{artf}=[minimum size = 5mm]
%     
%     \node[artf] (drat) {DRAT certificate};
%     \node[tool] (gratgen) [below of = drat] {gratgen};
%     \node[artf] (cnf) [left of = gratgen, anchor=east] {CNF formula};
%     \node[artf] (grat) [below of = gratgen] {GRAT certificate};
%     \node[tool] (gratchk) [below of = grat] {gratchk};
%     \node[artf] (res) [right of = gratchk, anchor=west] {s UNSAT / s ERROR};
%     
%   
%     \draw[->] (cnf) to (gratgen);
%     \draw[->] (drat) to (gratgen);
%     \draw[->] (gratgen) to (grat);
%     \draw[->] (grat) to (gratchk);
%     \draw[->] (gratchk) to (res);
%     \draw[->] (cnf) |- (gratchk);
%     
%     \begin{uncoverenv}<3->
%     \node[rectangle callout, draw, callout absolute pointer=(gratgen.east), above right = of gratgen, align=left] {Expensive \\ Complex \\ Highly optimized};
%     \end{uncoverenv}
%     \node<4->[rectangle callout, draw, callout absolute pointer=(gratchk.east), above right = of gratchk, align=left] {Cheap \\ Simple \\ Formally verified};
%     
%   \end{tikzpicture}  
%   \end{onlyenv}
%   
% # Moving Complexity to gratgen
%   *<+-> Checking direction
%     * gratgen works in backwards mode, trims certificate
%     * gratchk works in forward mode
%   *<+-> Unit propagation
%     * gratgen outputs unit/conflict clauses
%     * gratchk only checks that clauses are unit/conflict
%     * Cruz-Filipe et al. [TACAS'17]
%   *<+-> RAT candidate counts
%     * gratgen gathers RAT candidates, and outputs RAT counts
%     * gratchk only checks that candidates are exhaustive
%       * maintaining candidate lists based on RAT counts
%     
% # gratgen
%   *<+-> Initially: Clone of drat-trim's backwards mode
%     * Uses two-watched literals, core-first unit propagation
%     * Implemented in \CC11
%   *<+->+: Separate watchlists: Faster core-first propagation
%   *<+->+: RAT-run heuristics: Faster RAT-candidate finding
%   *<+->+: Optional multithreading: Parallel checking of lemmas
%     
% # gratchk
%   *<+-> Streaming for lower memory footprint
%     * Unit/conflict clauses and deletion information not kept in memory
%   *<+-> Formally verified with Isabelle/HOL
% %     * LCF-style prover, reducing trust to small logical inference kernel
% %   * Uses imperative data structures
% %     * quite fast for verified program!
%   *<+-> Verification uses stepwise refinement approach
%     
% # Stepwise Refinement    
%   *<+-> Abstract description of algorithm
%     * Easy to verify
%   *<+-> Refine towards implementation
%     * Small, independent, correctness preserving steps
%   *<+-> Goal: Modularization of proofs
%     * Efficient implementation does not affect abstract correctness proof
%     * Making big proofs manageable in first place
%   *<+-> Isabelle Refinement Framework
%     * Powerful tool support for refinement proofs
%     * Library of verified (imperative) data structures
%     
% #! Refinement of gratchk --- Checker State
%   \begin{uncoverenv}<1->
%   Abstract: Using Sets and Functions
%   \begin{lstlisting}
%     literal = Pos var | Neg var
%     clause = literal set  
%     clausemap = (id => clause option) \<times> (literal => id set option)
%     state = id \<times> clausemap \<times> (var => bool option)
%   \end{lstlisting}
%   \end{uncoverenv}
%     
%   \begin{uncoverenv}<2->
%   Refinement \#1: Using iterators for clauses, lists for RAT candidates  
%   \begin{lstlisting}
%     clausemap1 = (id => $\color{red}\textsf{it}$) \<times> (literal => id $\color{red}\textsf{list}$ option)
%     state1 = id \<times> clausemap1 \<times> (var => bool option)
%   \end{lstlisting}
%   \end{uncoverenv}
%     
%   \begin{uncoverenv}<3->
%   Refinement \#2: Iterators as indexes into array, literals as integers, functions as arrays
%   \begin{lstlisting}
%     clausedb2 = $\color{red}\textsf{int array}$
%     clausemap2 = $\color{red}\textsf{nat array}$ \<times> id list option $\color{red}\textsf{array}$
%     state2 = id \<times> clausemap2 \<times> $\color{red}\textsf{nat array}$
%   \end{lstlisting}
%   \end{uncoverenv}
%     
% #! Final Correctness Statement    
%   \begin{uncoverenv}<+->
%   \begin{lstlisting}
%       <DBi |->$_a$ DB> 
%         verify_unsat_impl DBi F_end it prf
%       <\<lambda>r. DBi |->$_a$ DB * \<up>(r ==> /@formula_unsat_spec@/ DB F_end)>
%   \end{lstlisting}
%   \end{uncoverenv}
% 
%   \begin{uncoverenv}<+->
%   where
%   \begin{lstlisting}
%     /@formula_unsat_spec@/ :: "int list => nat => bool"
%     /@formula_unsat_spec@/ DB F_end == (
%       let lst = tl (take F_end DB) in
%         1 < F_end \<and> F_end \<le> length DB \<and> last lst = 0 
%         \<and> (\<nexists>\<sigma>. /@assn_consistent@/ \<sigma> \<and> (\<forall>C\<in>set (tokenize 0 lst). \<exists>l\<in>set C. \<sigma> l)))
%   \end{lstlisting}
%   \end{uncoverenv}
% 
%   \begin{uncoverenv}<+->
%   where
%   \begin{lstlisting}
%     /@assn_consistent@/ :: "(int => bool) => bool"
%     "/@assn_consistent@/ \<sigma> = (\<forall>x. x\<noteq>0 ==> \<not> \<sigma> (-x) = \<sigma> x)"
%   \end{lstlisting}
%   \end{uncoverenv}
% 
%   \begin{uncoverenv}<+->
%   Trusted base: Hoare triples + standard list functions  
%   \end{uncoverenv}
%     
% #! Generating gratchk
%   *<+-> Export from Isabelle to Standard/ML 
%     * Trusted: Isabelle code generator (extended pretty printing)
%   *<+-> Add parser and command line interface
%     * Trusted: Parsing DIMACS to int array (<100 LOC, simple)
%       * Standard ML detects overflows by design
%     * Trusted: CLI (straightforward)
%   *<+-> Compile with MLton
%     * Trusted: Compiler
% 
% \newcommand{\annot}[1]{{\footnotesize\color{red}{#1}}}
% 
% \newcommand{\uc}[2]{\uncover<#1->{#2}}  
% 
% 
% # Benchmark Results
%   \begin{uncoverenv}<1->
%   Using 2016 SAT competition main track:
%     * cmsat: 110 UNSAT, 64 SAT  (many RAT lemmas)
%     * riss6: 128 UNSAT          (Silver medalist, but only RUP) 
%   \end{uncoverenv}  
% 
%   \vfill
%     
%     
%   \begin{uncoverenv}<2->
%   \begin{tabular}{c|c|c|c|c|}
%             & lrat*                             & drat-trim                         & grat/1          & grat/8                    \\\hline
%       cmsat & \uc{3}{51h \annot{2xT/O, 1xSEGF}} & \uc{4}{42h \annot{2xT/O, 1xSEGF}} & \uc{5}{17h}     & \uc{6}{7h               } \\\hline
%       riss6 & \uc{3}{42h}                       & \uc{4}{30h                      } & \uc{5}{26h}     & \uc{6}{14h \annot{1xOOM}} \\\hline\hline
%     $\Sigma$& \uc{3}{93h}                       & \uc{4}{72h                      } & \uc{5}{44h}     & \uc{6}{21h              } \\\hline
%   \end{tabular}
%   {\tiny\vspace*{1em}}
%   
%   {\tiny Wall-clock times, excluded the 3 certificates that drat-trim failed on\\
%   \begin{tabular}{ll}
%   {\color{red} T/O} &--- Timeout after 20.000s \\
%   {\color{red} SEGF} &--- crashed with SIGSEGV \\
%   {\color{red} OOM} &--- out of memory (128GiB) \\
%    {*} &--- Incremental version, [Heule et al., ITP'17] \\
%   \end{tabular}
%   }
%   \end{uncoverenv}  
%   
%   \vfill
%   \begin{uncoverenv}<7->
%   gratchk time: 12\% (cmsat: 6\%, riss6: 16\%)
%   \end{uncoverenv}  
%   
%   \vfill
%   
%   \begin{uncoverenv}<8->
%   SAT mode: cmsat: 40s
%   \end{uncoverenv}  
% 
% # Conclusions
%   Fast SAT solver certification tool
%   * SAT and UNSAT certificates
%   * Strong formal correctness guarantees
%     * formally verified down to int array representing formula
%   * Available under BSD-style license
%     
%   \vfill 
%   \center \url{http://www21.in.tum.de/~lammich/grat/}
%     
%   \vfill
%   Tool Paper on SAT'2017
%     
%     
%     
% %   
% % # Enriched Certificates
% %   * Idea: Replace unit-propagation by unit checking
% %     * Cruz-Filipe et al. [TACAS'17]: For (weaker) DRUP certificates
% %   * During certificate checking: $(F \wedge \neg C)^{\textrm u} = \lightning$
% %     * Output list of clauses in order they become unit
% %     * Output conflict clause
% %     * Trim certificate
% %   * Then run 2nd checker
% %     * Against original formula, using enriched certificate
% %     * Only check that clauses are actually unit/conflict
% %     * Simple + cheap $\implies$ amenable to formal verification
% %     
%     
    
\end{document}




