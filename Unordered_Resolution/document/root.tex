\documentclass[11pt,a4paper]{article}
\usepackage[utf8]{inputenc}
\usepackage{isabelle,isabellesym}
\usepackage[left=2.25cm,right=2.25cm,top=2.25cm,bottom=2.75cm]{geometry}

% this should be the last package used!
\usepackage{pdfsetup}

% urls in roman style, theory text in math-similar italics
\urlstyle{rm}
\isabellestyle{it}

\begin{document}

\title{Unordered Resolution -- with Unification Theorem}
\author{Anders Schlichtkrull}
\maketitle
\begin{abstract}
This is an Isabelle formalization of unordered resolution for first-order logic. Most of it is loaded from the Archive of Formal Proofs \cite{afp}. The only exception is a formal proof of the unification theorem. It is proven here by loading it from a theory in the IsaFoR project \cite{isafor} and doing conversions between the terms of the two theories. The assumption in the unification locale can then be instantiated, and we obtain the completeness theorem from the locale.
\end{abstract}

\tableofcontents

% sane default for proof documents
\parindent 0pt\parskip 0.5ex

% generated text of all theories
\input{session}

% optional bibliography
\bibliographystyle{abbrv}
\bibliography{root}

\end{document}

%%% Local Variables:
%%% mode: latex
%%% TeX-master: t
%%% End:
