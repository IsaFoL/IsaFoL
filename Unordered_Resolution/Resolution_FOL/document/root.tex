\documentclass[11pt,a4paper]{article}
\usepackage[utf8]{inputenc}
\usepackage{isabelle,isabellesym}

\usepackage{amssymb}
\usepackage{amstext}

% this should be the last package used!
\usepackage{pdfsetup}

% urls in roman style, theory text in math-similar italics
\urlstyle{rm}
\isabellestyle{it}

\newcommand{\dash}{\ensuremath{\text{-}}}
\newcommand{\colo}{\text{:}}
\newcommand{\metaq}{{\textstyle\bigwedge}}
\newcommand{\mapstox}{\text{$\,\mapsto$}}

\newcommand{\subls}{\,\, \raisebox{-0.1ex}{\scalebox{1.35}{$\cdot$}}{\raisebox{-0.3ex}{$_{\hspace{-0.4ex}\mathrm{ls}}\,$}}}
\newcommand{\subl}{\,\, \raisebox{-0.1ex}{\scalebox{1.35}{$\cdot$}}{\raisebox{-0.3ex}{$_{\hspace{-0.4ex}\mathrm{l}}\,$}}}
\newcommand{\varsls}{vars_{\mathrm{ls}}}
\newcommand{\unifierls}{unifier_{\hspace{0.15ex}\mathrm{ls}}}
\newcommand{\mguls}{mgu_{\mathrm{ls}}}
\newcommand{\groundl}{ground_{\mathrm{l}}}
\newcommand{\groundls}{ground_{\mathrm{ls}}}
\newcommand{\instanceofls}{instance\dash of_{\mathrm{ls}}}
\newcommand{\evalt}{eval_{\hspace{0.15ex}\mathrm{t}}}
\newcommand{\evalts}{eval_{\mathrm{ts}}}
\newcommand{\evall}{eval_{\mathrm{l}}}
\newcommand{\evalc}{eval_{\mathrm{c}}}
\newcommand{\evalcs}{eval_{\mathrm{cs}}}
\newcommand{\falsifiesl}{falsifies_{\mathrm{l}}}
\newcommand{\falsifiesg}{falsifies_{\mathrm{g}}}
\newcommand{\falsifiesc}{falsifies_{\mathrm{c}}}
\newcommand{\compl}{{\mathrm{c}}}
\newcommand{\compls}{{\mathrm{C}}}
\newcommand{\Cone}{C_\mathrm{1}}
\newcommand{\Ctwo}{C_\mathrm{2}}
\newcommand{\Bone}{B_\mathrm{1}}
\newcommand{\Btwo}{B_\mathrm{2}}
\newcommand{\Lone}{L_\mathrm{1}}
\newcommand{\Ltwo}{L_\mathrm{2}}
\newcommand{\stdone}{std_\mathrm{1}}
\newcommand{\stdtwo}{std_\mathrm{2}}

\begin{document}

\title{The Resolution Calculus for First-Order Logic}
\author{Anders Schlichtkrull}
\maketitle
\begin{abstract}
This theory is a formalization of the resolution calculus for first-order logic. It is proven sound and complete. The soundness proof uses the substitution lemma, which shows a correspondence between substitutions and updates to an environment. The completeness proof uses semantic trees, i.e. trees whose paths are partial Herbrand interpretations. It employs Herbrand's theorem in a formulation which states that an unsatisfiable set of clauses has a finite closed semantic tree. It also uses the lifting lemma which lifts resolution derivation steps from the ground world up to the first-order world. The theory is presented in a paper at the International Conference on Interactive Theorem Proving \cite{schlichtkrull} and an earlier version in an MSc thesis \cite{thesis}. It mostly follows textbooks by Ben-Ari \cite{ben-ari}, Chang and Lee \cite{chang}, and Leitsch \cite{leitsch}. The theory is part of the IsaFoL project \cite{isafol}.
\end{abstract}

\tableofcontents

% sane default for proof documents
\parindent 0pt\parskip 0.5ex

% generated text of all theories
\input{session}

% optional bibliography
\bibliographystyle{abbrv}
\bibliography{root}

\end{document}

%%% Local Variables:
%%% mode: latex
%%% TeX-master: t
%%% End:
